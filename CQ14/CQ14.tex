\documentclass[a4paper,12pt]{article}	% тип документа

\usepackage[a4paper,top=1.3cm,bottom=2cm,left=1.5cm,right=1.5cm,marginparwidth=0.75cm]{geometry} % field settings

\usepackage[T2A]{fontenc}		% кодировка
\usepackage[utf8]{inputenc}		% кодировка исходного текста
\usepackage[english,russian]{babel}	% локализация и переносы
\usepackage{indentfirst}

%Piece of code
\usepackage{listings}
\usepackage{xcolor}
\lstset
{
    language=C++,
    backgroundcolor=\color{black!4}, % set backgroundcolor
    basicstyle=\footnotesize,% basic font setting
}

%Drawings
\usepackage{graphicx}

\usepackage{wrapfig}

\usepackage{multirow}

\usepackage{float}

\usepackage{wasysym}

\usepackage[T1]{fontenc}
\usepackage{titlesec}

\setlength{\parindent}{3ex}

% Literature
\addto\captionsrussian{\def\refname{Литература.}}

%Header
\title{
	\center{\textbf{Контрольные вопросы. Задание 13.}}
	}


\begin{document}	% the beginning of the document

\maketitle

\section{В каких ситуациях используются типы std::pair и std::tuple?}

	std::pair удобен для хранения пар ключ-значение. При этом элементы могут быть разных типов. Хранение таких пар необходимо, например, при использовании ассоциативных массивов). std::pair используется 1) классами map, multimap и т.д.; 2) функциями, возвращающими два значения). Т.о., std::pair используется всюду, где два значения необходимо интерпретировать как одно целое.
	
	std::tuple (кортеж) расширяет концепцию пары на произвольное количество элементов (с помощью вариативных шаблонов), т.е. кортежи -- неоднородные списки элементов типы которых задаются или выводятся во время компиляции. При этом кортеж не является обычным контейнерным классом, в котором можно осуществлять обход элементов.

\section{Когда следует использовать контейнер std::array?}

	std::array следует использовать для хранения массива однотипных данных фиксированного размера (т.е. размер массива должен быть известен на этапе компиляции) с произвольным доступом при необходимости относительно быстрой работы с данными (память для данных выделяется в "быстром" стеке). В целом, std::array, формально говоря, является более безопасной заменой встроенного массива []-array (например, метод array.at(index) является более безопасным, чем операция array[index]).

\section{Когда следует использовать контейнер std::vector?}

	std::vector управляет однотипными элементами, хранящимися в динамическом массиве (аналог new[]-массива), обеспечивая произвольный доступ к элементам. Добавление и удаление элементов происходит в конце массива и выполняется очень быстро. Но вставка в начало/в середину производится достаточно долго (О(n)?). Вектор "следует использовать по умолчанию", когда заранее неизвестно количество данных и не определено время работы структуры. Он соблюдает требования идиомы RAII и основы ООП. Кроме того, в векторе данные хранятся непрерывно (благодаря чему можно использовать арифметику указателей, что уже не справедливо, например, для дека.

\section{Когда следует использовать контейнер std::deque?}

	std::deque (двусторонняя очередь, дек) (является "двунаправленным" динамическим массивом), следует использовать, когда необходимо большое количество добавлений элементов (однотипных) в начало или конец контейнера (они будут выполняться за О(1) благодаря "страничной" структуре памяти). При этом вставка элемента в середину очереди может потребовать значительно больше времени.

\section{Когда следует использовать контейнер std::list?}

	std::list -- двусвязный список (возможно итерирование в двух направлениях), который следует использовать, когда есть необходимость в быстром осуществлении вставки или удалении элементов (однотипных) с любой позиции (за О(1)). В то же время std::list не поддерживает произвольный доступ (лишь при итерировании по контейнеру -- за линейное время).

\section{Когда следует использовать контейнер std::forward\_list?}

	std::forward\_list -- односвязный список (в отличие от двусвязного каждый элемент ссылается только на следующий, а последний -- на nullptr), являющийся сильно ограниченным (не реализованы даже push\_back() и size()) и в то же время очень экономным по памяти контейнером, чем и может быть обусловлена области использования односвязных списков. Работа с данным контейнером осуществялется с помощью специальных функций-членов.

\section{Какие адаптеры контейнеров есть в стандартной библиотеке?}

	Адаптеры контейнеров STL приспосабливают стандартные контейнеры для особых целей. В соответствии с этим существуют такие адаптеры как:
	
	\begin{itemize}
	
		\item Стек (LIFO).
		
		\item Очередь (FIFO).
		
		\item Очередь с приоритетами (FIFO + sort).
	
	\end{itemize}

\section{Когда следует использовать контейнер circular buffer из Boost?}

	circular buffer из Boost следует использовать для хранения истории фиксированной длины, когда непрерывно поступают новые данные: после заполнения всех ячеек контейнера, новые данные начинают перезаписываться с начала (что делает функционирование циклического буфера довольно быстрым). Часто circular buffer справляется с задачами быстрее, чем std::list или std::deque.

\section{Почему контейнер circular buffer из Boost не может войти в стандарт?}

	circular buffer из Boost не может войти в стандарт, т.к. 

\section{Какие типы данных для работы с многомерными массивами вы можете назвать?}

	Данные в многомерных массивах можно хранить, используя:
	
	\begin{itemize}
	
		\item boost::multi\_array;
		
		\item std::valarray;
		
		\item Кроме того можно составить контейнер из контейнеров (или массив из массивов) с помощью?
		
		\begin{itemize}
		
			\item std::vector;
		
			\item std::array;
		
			\item std::deque;	
			
			\item встроенные []-массивы.

		\end{itemize}	
	\end{itemize}
	
	Кроме того, при работе с многомерными массивами (и одномерными) оказываются полезными	итераторы и указатели.
	
\newpage


\addcontentsline{toc}{section}{Литература}
 
	\begin{thebibliography}{}
	
		\bibitem{litlink1} Конспект семинара. Макаров И.С.
		\bibitem{litlink2} Standard library. Dzhosattis N.
		\bibitem{litlink3} chrome-extension://efaidnbmnnnibpcajpcglclefindmkaj/https://open-std.org/JTC1/SC22/WG21/docs/papers/2017/p0059r3.pdf
		\bibitem{litlink4} https://www.codeproject.com/Articles/1185449/Performance-of-a-Circular-Buffer-vs-Vector-Deque-a
		
	\end{thebibliography}


\end{document} % end of the document
